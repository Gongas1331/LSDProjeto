\documentclass[11pt,twoside,a4paper]{report}
\usepackage[DETI,newLogo]{uaThesis}

\def\ThesisYear{2012}

% optional packages
\usepackage[portuguese]{babel}
\usepackage{hyperref}
\usepackage{amsmath}
\usepackage{amssymb}

\begin{document}

%
% Tables of contents, of figures, ...
%

\pagenumbering{roman}
\tableofcontents


\pagenumbering{arabic}
\chapter{Introdução}

	Este projeto tem como objetivo a modelação em VHDL e a implementação na FPGA um sistema digital que simulasse uma máquina de lavagem de roupa com um botão de start/stop e outro de reset, tendo esta três programas possíveis que são compostos de 4 tarefas diferentes.

	A máquina permite também observar ao longo do seu funcionamento se esta se encontra a meio de um programa, em que tarefa se encontra, e quanto tempo falta para esta acabar.


\chapter{Arquitetura}

 O sistema é constituído por seis entradas: \textbf{CLOCK\_50}, \textbf{KEY0}, \textbf{KEY1}, \textbf{SW0}, \textbf{SW1}, \textbf{SW2}, \textbf{SW3} e \textbf{SW4}. O botão \textbf{KEY0} tem como função reiniciar a máquina de lavar. O botão \textbf{KEY1} serve para começar/parar/retomar o funcionamento dos programas da máquina. Os interruptores \textbf{SW1}, \textbf{SW2}, \textbf{SW3} servem para selecionar os programas 1, 2 e 3, respetivamente. O interruptor SW0 simula a porta da máquina e o interruptor \textbf{SW4} ativa o modo deferido da máquina. 
 
 O sistema é constituído também por 17 saídas: \textbf{LEDG7}, \textbf{LEDG3}, \textbf{LEDG2}, \textbf{LEDG1}, \textbf{LEDG0}, \textbf{LEDR4}, \textbf{LEDR3}, \textbf{LEDR2}, \textbf{LEDR1}, \textbf{LEDR0}, \textbf{HEX6}, \textbf{HEX5}, \textbf{HEX4}, \textbf{HEX3}, \textbf{HEX2}, \textbf{HEX1} e \textbf{HEX0}. 
 
 Os leds verdes \textbf{LEDG7}, \textbf{LEDG3}, \textbf{LEDG2}, \textbf{LEDG1}, \textbf{LEDG0} ativam, respetivamente, se a porta da máquina estiver aberta, se a função de spin estiver a ser executada, se a bomba de água estiver ativa, se a máquina estiver a enxaguar, e se a válvula de água. 
 
 Nos leds vermelhos, \textbf{LEDR4}, \textbf{LEDR3}, \textbf{LEDR2}, \textbf{LEDR1} e \textbf{LEDR0}, o \textbf{LEDR4} ativa quando o modo diferido está ligado, o \textbf{LEDR0} liga enquanto a máquina executa um programa, enquanto os restantes leds vermelhos mostram que programa está selecionado, correspondendo o \textbf{LEDR1} ao programa 1, o \textbf{LEDR2} ao programa 2 e o \textbf{LEDR3} ao programa 3.
 
 Nos displays hexadecimais textbf{HEX6}, \textbf{HEX5}, \textbf{HEX4}, \textbf{HEX3}, \textbf{HEX2}, \textbf{HEX1} e \textbf{HEX0} pode-se observar o valor do tempo total do programa antes deste ser iniciado, ou o tempo restante para acabar uma tarefa nos displays \textbf{HEX5} e \textbf{HEX4}, o display \textbf{HEX6} mostra um igual(=) quando a máquina está pausada, os displays \textbf{HEX3} e \textbf{HEX2} mostram "dF" enquanto se espera a ativação da máquina no modo diferido, e os displays \textbf{HEX1} e \textbf{HEX0} mostram que programa está selecionado.
 
 
\chapter{Manual do Utilizador}

\chapter{Implementação}

Apresentamos de seguida, uma lista das opções suportadas.
\begin{itemize}
  \item \verb+oldLogo+: usa o ``antigo'' logotipo da Universidade de Aveiro.
  \item \verb+newLogo+: usa o ``novo'' logotipo da Universidade de Aveiro.
  \item \verb+final+: \textbf{n\~ao escreve} o texto ``documento provis\'orio'' na capa: al\'em
        disso, todas as marcas que assinalam linhas demasiado compridas s\~ao eliminadas.
  \item \verb+DETI+, \verb+DM+, \verb+DF+: para teses escritas por alunos dos departamentos de
        electr\'onica, telecomunica\c c\~oes e inform\'atica, de matem\'atica, e de f\'\i sica.
        \'E muito f\'acil incluir uma op\c c\~ao para um outro departamentos editando o
        ficheiro \verb+uaThesis.sty+.
\end{itemize}

\chapter{Conclusão}

Este projeto foi aquele que nos chamou mais à atenção, sendo que nos pareceu o mais interessante de resolver.

\chapter{Contribuição}

\textbf{\Large Gonçalo Cunha} :

\textbf{\Large Anderson Lourenço} :

\end{document}
