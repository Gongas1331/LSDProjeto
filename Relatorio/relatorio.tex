

%
% Tables of contents, of figures, ...
%

\pagenumbering{roman}
\tableofcontents


% The chapters (usually written using the isolatin font encoding ...)

\pagenumbering{arabic}
\chapter{Introdução}

	Este projeto tem como objetivo a modelação em VHDL e a implementação na FPGA um sistema digital que simulasse uma máquina de lavagem de roupa com um botão de start/stop e outro de reset, tendo esta três programas possíveis que são compostos de 4 tarefas diferentes.

	A máquina permite também observar ao longo do seu funcionamento se esta se encontra a meio de um programa, em que tarefa se encontra, e quanto tempo falta para esta acabar.

\chapter{Manual do Utilizador}

	

\chapter{Arquitetura}

O sistema é constituído por seis entradas: CLOCK\_50, KEY0, KEY1, SW0, SW1, SW2, SW3 e SW4. 
 O botão KEY0 tem como função reiniciar a máquina de lavar. O botão KEY1 serve para começar/parar/retomar o funcionamento dos programas da máquina. Os interruptores SW1, SW2 e SW3 servem para selecionar os programas 1, 2 e 3, respetivamente. O interruptor SW0 simula a porta da máquina e o interruptor SW4 ativa o modo deferido da máquina. 

\chapter{Implementação}

Apresentamos de seguida, uma lista das opções suportadas.
\begin{itemize}
  \item \verb+oldLogo+: usa o ``antigo'' logotipo da Universidade de Aveiro.
  \item \verb+newLogo+: usa o ``novo'' logotipo da Universidade de Aveiro.
  \item \verb+final+: \textbf{n\~ao escreve} o texto ``documento provis\'orio'' na capa: al\'em
        disso, todas as marcas que assinalam linhas demasiado compridas s\~ao eliminadas.
  \item \verb+DETI+, \verb+DM+, \verb+DF+: para teses escritas por alunos dos departamentos de
        electr\'onica, telecomunica\c c\~oes e inform\'atica, de matem\'atica, e de f\'\i sica.
        \'E muito f\'acil incluir uma op\c c\~ao para um outro departamentos editando o
        ficheiro \verb+uaThesis.sty+.
\end{itemize}

\chapter{Conclusão}

Este projeto foi aquele que nos chamou mais à atenção, sendo que nos pareceu o mais útil a nível de referência futura.

\chapter{Contribuição}

\textbf{\Large Gonçalo Cunha} : tudo

\textbf{\Large Anderson Lourenço} : nada

\end{document}
